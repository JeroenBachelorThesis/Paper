\subsection{Object pooling}
\label{sec:object-pooling}

In chapter \ref{sec:object-reuse} we've discussed a specific approach at minimizing the amount of objects created. In this section we will take a look at a more general implementation of this idea. While nodes and tokens are at the very core of the Roslyn code base, they're not the only objects we'll create. 

When creating a lexer, one of the major tasks you're performing is reading source code and creating strings. If you would intermittently concatenate strings, this would increase memory pressure by a lot: every concatenation using \verb|+| creates a new string but you might only be interested in the result\footnote{https://msdn.microsoft.com/en-us/library/2839d5h5(v=vs.110).aspx}. 
However creating a new \verb|StringBuilder| object every time you want to concatenate strings is an expensive operation as well: since a lot of string concatenation is done, a lot of distinct \verb|StringBuilder| instances will be created and you again have the issue as described before. The solution here is to take pooling one step further: every type can be pooled through the \verb|ObjectPool<T>| class.\parencite{Warren2014} This provides a generic implementation of the object pooling pattern and will be used through intermediate classes.

An accurate description of this class' workings can be found in its documentation\footnote{\url{https://github.com/dotnet/roslyn/blob/b908b05b41d3adc3b5e81f8cf2d0055c13e4a1f6/src/Compilers/Core/SharedCollections/ObjectPool\%601.cs}}:

\begin{quotation}
Generic implementation of object pooling pattern with predefined pool size limit. The main
purpose is that limited number of frequently used objects can be kept in the pool for
further recycling.

Notes: 

1) it is not the goal to keep all returned objects. Pool is not meant for storage. If there
   is no space in the pool, extra returned objects will be dropped.

2) it is implied that if object was obtained from a pool, the caller will return it back in
   a relatively short time. Keeping checked out objects for long durations is ok, but 
   reduces usefulness of pooling. Just new up your own.

Not returning objects to the pool in not detrimental to the pool's work, but is a bad practice. 
Rationale: 
   If there is no intent for reusing the object, do not use pool - just use "new". 
\end{quotation}

An interesting addition here is the implementation of \verb|SharedPools|. As the name indicates, this is a class that manages the sharing of object pools. There are two ways to use this class:

\begin{itemize}
\item As a single, standalone pool

This does not share the pool but instead just makes it easier to create your object pools. For this purpose you can choose between a small pool (20 elements) and a big pool (100 elements). For example in the case of a small pool you call

\lstset{style=csharp, caption={SharedPools.Default<T>}}
\begin{lstlisting}
public static ObjectPool<T> Default<T>() 
																where T : class, 
																					new()
{
	return DefaultNormalPool<T>.Instance;
}
\end{lstlisting}

This method, in turn, will lazily create a new pool the first time a pool of that type is requested. This pool will defer to the beforementioned \verb|ObjectPool| implementation:

\lstset{style=csharp, caption={SharedPools.DefaultNormalPool<T>}}
\begin{lstlisting}
private static class DefaultNormalPool<T> 
																where T : class, 
																					new()
{
	public static readonly ObjectPool<T> Instance = 
		new ObjectPool<T>(() => new T(), 20);
}
\end{lstlisting}

\item As a shared pool

Taking the entire idea another step further, you can also share the pools themselves. This is useful to share data inside the pools: if in one parsing session you create a string "using" then it would be nice to re-use that in another parsing session. This follows the same idea of the statically shared \verb|TextKeyedCache| pool as we talked about in section \ref{sec:re-use-tokens} but expands this usage across all types due to its generic nature. Here again we see two separate usages:

\subitem As a pre-defined pool

Inside the \verb|SharedPools| we have a few pre-defined pools for general use cases. One of those is \verb|StringIgnoreCaseHashSet| which provides a hashset pool that compares its elements case-insensitively.

\lstset{style=csharp, caption={SharedPools.StringIgnoreCaseHashSet}}
\begin{lstlisting}
public static readonly 
	ObjectPool<HashSet<string>> StringIgnoreCaseHashSet =
	new ObjectPool<HashSet<string>>(
		() => new HashSet<string>(
							StringComparer.OrdinalIgnoreCase), 20);
\end{lstlisting}

\subitem As a pre-defined type

There are also specific types that provide some type-specific behaviour. An example of this is the \verb|StringBuilderPool|:

\lstset{style=csharp, caption={StringBuilderPool}}
\begin{lstlisting}
internal static class StringBuilderPool
{
	public static StringBuilder Allocate()
	{
		return SharedPools.Default<StringBuilder>()
										  .AllocateAndClear();
	}
	
	public static void Free(StringBuilder builder)
	{
		SharedPools.Default<StringBuilder>()
							 .ClearAndFree(builder);
	}
	
	public static string ReturnAndFree(
													StringBuilder builder)
	{
		SharedPools.Default<StringBuilder>()
							 .ForgetTrackedObject(builder);
		return builder.ToString();
	}
}
\end{lstlisting}

\end{itemize}