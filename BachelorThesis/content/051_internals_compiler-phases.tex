\section{Compiler phases}
\label{sec:compiler-phases}

A \gls{compiler} is a complicated application and consists of many different services. Some of these are meant to help developers write their own tools (which is what this paper is about) while some of those services make up the internal tooling. During the compilation process we can identify 4 phases\footnote{\url{https://github.com/dotnet/roslyn/wiki/Roslyn\%20Overview\#exposing-the-compiler-apis}} where we go from the original source code to the eventual compilation into \gls{msil}. One important remark to make here is that not everything happens from start to finish together: symbols are created on-demand and as such, some symbols might never be materialized.

\subsection{Parsing phase}
\label{sec:parsing-phase}

This is the first phase and receives the source code. Here, the source is tokenized into syntactical constructs according to the C\# Language Specification. This process is performed by two important classes: \texttt{Lexer.cs} and \texttt{LanguageParser.cs}. Both services their function overlaps largely but the lexer works at a lower level than the \gls{parser}: while the lexer can give meaning to certain symbols (not to be confused with the symbols in section \ref{sec:declaration-phase}) such as operators, the parser is able to combine multiple tokens and construct larger syntactical constructs such as expressions.\parencite{Studencki2010}

\begin{minipage}{.95\linewidth}
The eventual result of the parsing phase will be a \gls{syntaxtree} which also means that every part of the source code is parsed in this step. This phase encompasses the concepts of lexical analysis (understanding individual words) and syntactic analysis (understanding sentences). It is in this phase that syntax errors will be caught and no further analysis on those nodes is required. The parser is able to handle invalid code through two approaches: \parencite{Ebeid2013}

\begin{itemize}
\item If it can make a guess as to what token should be expected, a missing token will be inserted and have its \texttt{IsMissing} property set
\item Otherwise it can also skip tokens and add them as trivia until it finds a location where parsing can continue
\end{itemize}
\end{minipage}

\subsection{Declaration phase}
\label{sec:declaration-phase}

Once the \gls{syntaxtree} has been constructed in the parsing phase, the declaration phase will add meaning to all the nodes. Imagine going from a blank \gls{syntaxtree} to a \gls{syntaxtree} that uses syntax highlighting: that happens here. Like the parsing phase, this step is executed for all tokens in the \gls{syntaxtree} and has as result a look-up table with the created symbols\footnote{\url{http://stackoverflow.com/a/10183455/1864167}}. In the next phase, these constructed symbols can be used further. This phase is together with the next one a part of the semantic analysis concept. This step will perform more complicated analysis by verifying the parsed symbols adhere to the C\# Language Specification.

\subsection{Binding phase}
\label{sec:binding-phase}

The binding phase will, on-demand, bind symbols to identifiers in text. This is also part of the semantic analysis and results in the eventual semantic analysis. As a developer you will most often interact with this through the \texttt{semanticModel.GetSymbolInfo()} (and related) methods where we can retrieve the results. The previously mentioned method will allow us to retrieve a symbol from a given syntax node -- which is what's created in the declaration phase.

\subsection{Emit phase}
\label{sec:emit-phase}

Lastly, the emit phase will bundle everything in a module, wrap it in an assembly and emit to an outputstream together with debug information (\gls{pdb} files), XML documentation, etc. One such a result that can come from this is a class library on disk.