\section{Performance considerations}
\label{sec:performance}

Performance is an integral aspect of any application and as such Roslyn doesn't escape from it either. It's not just about compiling an assembly as fast as possible: that's just one of the usecases in which Roslyn is used. Think about other scenarios such as getting quick intellisense\footnote{Intellisense: context-aware hints as you type} or fluent syntax highlighting of text as you scroll but also affects other performance aspects such as the memory impact.

In this section we will look at a handful of techniques the Roslyn team uses to optimize the platform. These are general approaches that often apply their ideas in several areas of the codebase, sometimes through a common resource. This should indicate that these are often optimizations at a high level rather than very specific single-use optimizations. 

\subsection{Concurrency}
\label{sec:concurrency}

\subsection{Small nodes}
\label{sec:small-nodes}

\subsection{Object re-use}
\label{sec:object-reuse}

\subsection{Weak reference}
\label{sec:weak-references}

\subsection{Object pooling}
\label{sec:object-pooling}

\subsection{Specialized collections}
\label{sec:specialized-collections}

\subsection{LINQ}
\label{sec:linq}
















