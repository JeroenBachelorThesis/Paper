\subsection{Specialized collections}
\label{sec:specialized-collections}

Collections in the .NET framework are heavily optimized by themselves already but it is done so in a general manner: in the end, the .NET framework will be used by many different people so optimizations have to be weighed against a lot of different scenarios. By creating your own specialized collections, you have the possibility to cut corners that might not be applicable in your domain's scenario. In its crudest form: if you have a \texttt{null} check inside the collection's operations but you know for sure that that can never be \texttt{null} then that is an instruction you can skip. Another important reason to use your own custom collections is to formalize the specific behaviour of your domain that belongs to a collection. 

\subsubsection{IdentifierCollection}
\label{sec:spec-coll-identifiercollection}

An example of the latter can be found in the form the \texttt{IdentifierCollection}. This class acts as a wrapper around a \texttt{Dictionary} that maps a certain key to one or more alternative spellings of that key. While in essence nothing more than a mapper between key and value, it provides more intuitive methods to interact with such as \texttt{AddIdentifier}, \texttt{AddInitialSpelling} and \texttt{AddAdditionalSpelling}.
In this case, we can see a performance consideration in the way it stores the alternative identifiers: the backing \texttt{Dictionary} is actually mapping a \texttt{string} to an \texttt{object} instead of an array as one would expect. The reason this is done is to avoid having to allocate an array when there is just one alternative identifier -- now we can just define it as a \texttt{string} without the need for an array.

We can see this consideration reflected when we try to add a new spelling to a given key:

\lstset{style=csharp, caption={IdentifierCollection.AddAdditionalSpelling}}
\begin{lstlisting}
private void AddAdditionalSpelling(string identifier, 
																	 object value)
{
	// Had a mapping for it.  It will either map to a single 
	// spelling, or to a set of spellings.
	var strValue = value as string;
	if (strValue != null)
	{
		if (!string.Equals(identifier, 
											 strValue, 
											 StringComparison.Ordinal))
		{
			// We now have two spellings. Create a collection 
			// for that and map the name to it.
			_map[identifier] = new HashSet<string> 
				{ identifier, strValue };
		}
	}
	else
	{
		// We have multiple spellings already.
		var spellings = (HashSet<string>)value;

		// Note: the set will prevent duplicates.
		spellings.Add(identifier);
	}
}
\end{lstlisting}

Notice also the note on line 20: smart usage of existing collections reduces the complexity of our own code.

\subsubsection{CachingDictionary}
\label{sec:spec-coll-cachingdictionary}

Another implementation is the \texttt{CachingDictionary}. As its name indicates, it represents an in-memory caching mechanism based on the \texttt{Dictionary} collection. There is a very interesting implementation detail here: the backing dictionary is a \texttt{ConcurrentDictionary} meaning its exposed methods are \gls{threadsafe}. However, when the backing dictionary is full, it actually swaps that out for a normal \texttt{Dictionary} to allow for less-expensive read-operations without the thread-safety overhead.

\lstset{style=csharp, caption={CachingDictionary.IsNotFullyPopulatedMap}}
\begin{lstlisting}
private static bool IsNotFullyPopulatedMap
		(IDictionary<TKey,	ImmutableArray<TElement>> map)
{
	return 
		map == null || 
		map is 
		 ConcurrentDictionary<TKey, ImmutableArray<TElement>>;
}
\end{lstlisting}