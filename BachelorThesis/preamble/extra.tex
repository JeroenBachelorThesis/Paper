\usepackage[utf8]{inputenc}  % Accenten gebruiken in tekst (vb. é ipv \'e)
\usepackage{amsfonts}        % AMS math packages: extra wiskundige
\usepackage{amsmath}         %   symbolen (o.a. getallen-
\usepackage{amssymb}         %   verzamelingen N, R, Z, Q, etc.)
\usepackage[english]{babel}    % Taalinstellingen: woordsplitsingen,
                             %  commando's voor speciale karakters
                             %  ("dutch" voor NL)
\usepackage{csquotes}

\usepackage{eurosym}         % Euro-symbool €
\usepackage{geometry}
\usepackage{graphicx}        % Invoegen van tekeningen
\usepackage[pdftex,bookmarks=true]{hyperref}
                             % PDF krijgt klikbare links & verwijzingen,
                             %  inhoudstafel
\usepackage{listings}        % Broncode mooi opmaken
\usepackage{multirow}        % Tekst over verschillende cellen in tabellen
\usepackage{rotating}        % Tabellen en figuren roteren
% \usepackage{natbib}          % Betere bibliografiestijlen
\usepackage{fancyhdr}        % Pagina-opmaak met hoofd- en voettekst

\usepackage[T1]{fontenc}     % Ivm lettertypes
\usepackage{lmodern}
\usepackage{textcomp}

\usepackage{lipsum}          % Voor vultekst (lorem ipsum)

\usepackage[backend=bibtex, style=authoryear, bibencoding=ascii]{biblatex}			
\addbibresource{sample}
\usepackage{array}

\usepackage{listings}				 % Used to get pretty C# syntax highlighting
\usepackage{xcolor}