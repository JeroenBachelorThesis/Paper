\usepackage[utf8]{inputenc}  % Accenten gebruiken in tekst (vb. é ipv \'e)
\usepackage{amsfonts}        % AMS math packages: extra wiskundige
\usepackage{amsmath}         %   symbolen (o.a. getallen-
\usepackage{amssymb}         %   verzamelingen N, R, Z, Q, etc.)
\usepackage[english]{babel}    % Taalinstellingen: woordsplitsingen,
                             %  commando's voor speciale karakters
                             %  ("dutch" voor NL)
\usepackage{csquotes}

\usepackage{eurosym}         % Euro-symbool €
\usepackage{geometry}
\usepackage{graphicx}        % Invoegen van tekeningen
\graphicspath{ {img/} }
\usepackage[pdftex,bookmarks=true]{hyperref}
                             % PDF krijgt klikbare links & verwijzingen,
                             %  inhoudstafel
\usepackage{listings}        % Broncode mooi opmaken
\usepackage{multirow}        % Tekst over verschillende cellen in tabellen
\usepackage{rotating}        % Tabellen en figuren roteren
% \usepackage{natbib}          % Betere bibliografiestijlen
\usepackage{fancyhdr}        % Pagina-opmaak met hoofd- en voettekst

\usepackage[T1]{fontenc}     % Ivm lettertypes
\usepackage{lmodern}
\usepackage{textcomp}

\usepackage{epigraph}				% Adds epigraph support

\usepackage{lipsum}          % Voor vultekst (lorem ipsum)

\usepackage[backend=bibtex, style=authoryear, bibencoding=ascii]{biblatex}			
\addbibresource{sample}

\usepackage{float}					% Image positioning

\usepackage{array}

\usepackage{listings}				 % Used to get pretty C# syntax highlighting
\usepackage{xcolor}

\usepackage{glossaries}			% Glossaries. Duh.
\makeglossaries
%% Glossary

\newglossaryentry{managedlanguage}
{
	name={managed language},
	description={A language which offsets memory management from the user to a Garbage Collector}
}

\newacronym{msil}{MSIL}{Microsoft Intermediate Language}

\newglossaryentry{semanticmodel}
{
	name={semantic model},
	description={A model of the meaning attributed to symbols}
}

\newglossaryentry{syntax}
{
	name={syntax},
	description={A collection of plain-text tokens that make up source code}
}

\newglossaryentry{dataflow}
{
	name={data flow},
	description={The execution path of a piece of code}
}

\newglossaryentry{opensource}
{
	name={open-source},
	description={Source code that is freely available and which can be modified by the public},
	sort={opensource}
}

\newglossaryentry{bootstrap}
{
	name={bootstrap},
	description={A self-starting process. A compiler that compiles itself}
}

\newglossaryentry{gc}
{
	name={Garbage Collector},
	description={A service which allocates and frees memory}
}

\newglossaryentry{unittest}
{
	name={unit test},
	description={A self-contained automated test which has no interaction with external systems and tests a software execution path}
}

\newglossaryentry{compiler}
{
	name={compiler},
	description={A program that translates code from one language to another}
}

\newacronym{pr}{PR}{Pull Request}

\newglossaryentry{parser}
{
	name={parser},
	description={A service that parses source code into syntactical constructs}
}

\newacronym{ide}{IDE}{Integrated Development Environment}

\newglossaryentry{solution}
{
	name={solution},
	description={A solution is a set of code files and other resources that are used to build an application}
}

\newglossaryentry{compiletimeconstant}
{
	name={compile-time constant},
	description={A variable of which its value is guaranteed known when the compilation process starts}
}

\newglossaryentry{syntaxtree}
{
	name={syntax tree},
	description={A data structure which represents the hierarchy between syntactical constructs}
}

\newglossaryentry{metadata}
{
	name={metadata},
	description={Information about the data or action}
}

\newglossaryentry{invocation}
{
	name={invocation},
	description={Executing a code path by calling a function}
}












\usepackage{attachfile}			% Attachments