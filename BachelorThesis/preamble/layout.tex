% hoofdingen, enz.
\pagestyle{fancy}
% enkel hoofdstuktitel in hoofding, geen sectietitel (vermijd overlap)
\renewcommand{\sectionmark}[1]{}

% lijn, wordt gebruikt in titelpagina
\newcommand{\HRule}{\rule{\linewidth}{0.5mm}}

% Leeg blad
\newcommand{\emptypage}{
\newpage
\thispagestyle{empty}
\mbox{}
\newpage
}

% Gebruik een schreefloos lettertype ipv het "oubollig" uitziende
% Computer Modern
\renewcommand{\familydefault}{\sfdefault}

\lstdefinestyle{csharp}{
	language=[sharp]C, 
	frame=lr,
	numbers=left,
	stepnumber=1,
	numbersep=10pt,
	basicstyle=\ttfamily\small,
	xleftmargin=\parindent,
	keywordstyle=\color{blue},
  commentstyle=\itshape\color{green},
  stringstyle=\color{purple},
  identifierstyle=\color{gray},
	tabsize=2,
	captionpos=b,
	rulecolor=\color{blue!80!black}
}

% Commando voor invoegen Java-broncodebestanden (dank aan Niels Corneille)
% Gebruik: \codefragment{source/MijnKlasse.java}{Uitleg bij de code}
%\newcommand{\codefragment}[2]{ \lstset{
  %language=java,
  %breaklines=true,
  %float=th,
  %caption={#2},
  %basicstyle=\scriptsize,
  %frame=single,
  %extendedchars=\true
%}
%\lstinputlisting{#1}}